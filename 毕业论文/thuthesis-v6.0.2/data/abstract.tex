% !TeX root = ../main.tex

% 中英文摘要和关键字

\begin{abstract}
  小型机器人能量密度高、运动灵活,在救援、侦察等方面有着巨大应用潜力,同时也因其易于制造和调试的特点而备受研究者青睐。本文作者以仿生设计出发,模仿自然界的蜜袋鼯设计了一种兼具跳跃和滑翔功能的小型运动装置,致力于改善小型机器人跳跃过程中的飞行性能,达到延长跳跃距离、提高空中姿态稳定性的目的。

  本文的创新点主要有:
  \begin{itemize}
    \item 在当前世界最先进的跳跃机器人的研究基础上增加了滑翔机构;
    \item 通过降维设计降低了机构制造成本;
    \item 通过磁场来储存能量,提供跳跃所需爆发力矩;
    \item 设计了精巧的榫卯结构,使主控板与机身各部件可靠连接。
  \end{itemize}

  % 关键词用“英文逗号”分隔
  \thusetup{
    keywords = {小型机器人, 仿生设计, 跳跃, 滑翔, 降维设计},
  }
\end{abstract}

\begin{abstract*}
  Mini-robots enjoy high power density and agility, which grants them great potential in applications such as reconnaissance, search and rescue. They are also welcomed by researchers due to being easier to produce and debug. Utilizing bio-mimic design, the author of this article designed a palm-sized robot capable of jumping and gliding imitating sugar gliders, aiming to improve the flight performance of the jumping robot,prolong jumping distance as well as stable in-the-air attitude.

  The innovations of this article are:
  \begin{itemize}
    \item Added gliding feature to state-of-the-art research of jumping robots.
    \item Reduced manufacture cost by dimension reduction design.
    \item Storing energy in the magnetic field to provide explosive torque required by the jumping afterwards.
    \item Robustly connected the main controller board to the body with carefully designed mortise and tenon structure.
  \end{itemize}
  \thusetup{
    keywords* = {Mini-robot, bio-mimic, jump, glide, dimension reduction design},
  }
\end{abstract*}
